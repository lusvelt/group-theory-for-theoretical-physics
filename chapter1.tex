\chapter{Formal group theory and representations}
The goal of this chapter is to develop the abstract tools to deal with groups, and therefore it aims to be general.
The main references for this chapter and the following ones are \cite{Barone2004} and \cite{Tung1985}.


\section{Basic Concepts}

The choice of the basic concepts follows \cite{Barone2004}.

\begin{definition}[Group]
    A group $G$ is a set endowed with an internal operation $\cdot: G \times G \rightarrow G$, which satisfies the following properties:
    \begin{enumerate}
        \item associativity
        \begin{equation}
            \forall g_1, g_2, g_3 \in G, (g_1 \cdot g_2) \cdot g_3 = g_1 \cdot (g_2 \cdot g_3)
            \label{eq:group_associativity}
        \end{equation}
        \item existance of neutral element
        \begin{equation}
            \exists e \in G, \forall g \in G, g\cdot e = e\cdot g = g
            \label{eq:group_neutral}
        \end{equation}
        \item existance of inverse
        \begin{equation}
            \forall g \in G, \exists g^{-1} \in G, g \cdot g^{-1} = g^{-1} \cdot g = e
            \label{eq:group_inverse}
        \end{equation}
    \end{enumerate}
\end{definition}
If it is non-ambiguous, one can omit the multiplication symbol $g \cdot h = gh$.

\begin{proposition}
    The following properties hold:
    \begin{enumerate}
        \item The identity element $e \in G$ is unique. \label{prop:unique_neutral}
        \item The inverse $g^{-1} \in G$ of each group element $g \in G$ is unique. \label{prop:unique_inverse}
        \item The inverse of the inverse is the original element: $\forall g \in G, (g^{-1})^{-1} = g$ \label{prop:inv_of_inv}
        \item $\forall g_1, g_2 \in G, (g_1 g_2)^{-1} = g_2^{-1} g_1^{-1}$ \label{prop:reverse_inv}
    \end{enumerate}
\end{proposition}
\begin{proof}
    \begin{enumerate}
        \item Suppose we had two identity elements $e_1, e_2 \in G$.
        Then, by \eqref{eq:group_neutral}, we have
        \begin{equation*}
            e_1 = e_1 \cdot e_2 = e_2
        \end{equation*}

        \item Consider $g \in G$ and suppose it had to inverses $h_1, h_2 \in G$.
        This means that $h_2 = e \cdot h_2 = (h_1 \cdot g) \cdot h_2 = h_1 \cdot (g \cdot h_2) = h_1 \cdot e = h_1$, where we used associativity \eqref{eq:group_associativity}.

        \item Consider an element $g \in G$. We have $g^{-1} \cdot (g^{-1})^{-1} = e = g^{-1} \cdot g$. If we multiply by $g$ on the left, we get $(g^{-1})^{-1} = g$.

        \item Consider $g_1, g_2 \in G$. Then, by associativity, we have $(g_2^{-1} \cdot g_1^{-1}) \cdot (g_1 \cdot g_2) = g_2^{-1} \cdot (g_1^{-1} \cdot g_1) \cdot g_2 = g_2^{-1} \cdot g_2 = e$.
    \end{enumerate}
\end{proof}

\begin{remark}
    Since the inverse is unique, the equation $a \cdot x = b$ for $a, b, x \in G$ implies a unique solution $x = a^{-1} \cdot b$. The same holds for $y \cdot a = b$, which yields $y = b \cdot a^{-1}$.

    In particular, for $u, v \in G$, the equality $a \cdot u = a \cdot v$ implies $u = w$ since we can multiply both sides by $a^{-1}$ on the left. Analogously, the equality $u \cdot a = v \cdot a$ implies $u = v$ after multiplying by $a^{-1}$ on the right of both sides.
\end{remark}

From now on, we will denote the identity element as $\idop$, and if the group to which it belongs is ambiguous, we add a subscript $\idop_g \in G$.

\begin{definition}[Abelian group]
    A group $G$ is said to be \emph{abelian} if its operation is commutative.
    \begin{equation}
        \forall g_1, g_2 \in G, g_1 \cdot g_2 = g_2 \cdot g_1
        \label{eq:abelian_group}
    \end{equation}
\end{definition}

\begin{definition}[Subgroup]
    Suppose we have a group $(G, \cdot)$ and $H \subset G$. Then, we say that $H$ is a subgroup of $G$ if $H$ itself is a group (with identity element $\idop_H = \idop_G$),which is closed under the group operation
    \begin{equation*}
        \forall h_1, h_2 \in H, h_1 \cdot h_2 \in H
    \end{equation*}
\end{definition}

\begin{definition}[Conjugation]
    Let $G$ be a group and $a, b \in G$. We say that $a$ and $b$ are conjugate if there exists $g \in G$ such that
    \begin{equation*}
        b = g \cdot a \cdot g^{-1}
    \end{equation*}
\end{definition}
\begin{example}
    In the case of $G = GL(N)$, which is the set of $N \times N$ invertible matrices, two matrices are conjugate if they are related by similarity $B = P A P^{-1}$.
\end{example}

\begin{proposition}
    Conjugacy is an equivalence relation.
\end{proposition}
\begin{proof}
    \begin{enumerate}
        \item Reflexivity: $a = gag^{-1}$ when $g = \mathbb{I} \in G$
        \item Symmetry: assume $b = gag^{-1}$ for some $g \in G$. Then, if we multiply by $g^{-1}$ on the left and $g$ on the right both sides, we get $g^{-1}bg = a$
        \item Transitivity: assume $b = gag^{-1}$ and $c=hbh^{-1}$, then $c = (hg)ag^{-1}h^{-1} = (hg)a(hg)^{-1}$
    \end{enumerate}
\end{proof}

\begin{definition}[Conjugacy class]
    Given an element $a \in G$, we define the following equivalence class, called \emph{conjugacy class} of $a$
    \begin{equation*}
        Cl(a) = \{h \in G | \exists g \in G, a = g h g^{-1}\}
    \end{equation*}
\end{definition}
\begin{remark}
    If the group $G$ is abelian, all its conjugacy classes are singlets, since $a = ghg^{-1} = hgg^{-1} = h$.
\end{remark}

\begin{definition}[Invariant subgroup]
    Let $G$ be a group and $H \subset G$ a subgroup. We say that $H$ is an \emph{invariant subgroup} if it is closed under $G$-conjugacy, i.e.
    \begin{equation*}
        \forall g \in G, \forall h \in H, ghg^{-1} \in H
    \end{equation*}
\end{definition}
\begin{remark}
    The subgroups $\{\idop\}$ and $G$ itself are the trivial invariant subgroups.
\end{remark}
\begin{remark}
    All the subgroups of an abelian group are necessarily invariant subgroups, since conjugacy does not affect their elements.
\end{remark}

\begin{definition}[Simple and semi-simple group]
    A group $G$ is said to be \emph{simple} if it has no non-trivial invariant subgroups.
    It is said \emph{semi-simple} if it has no non-trivial \emph{abelian} invariant subgroups.
\end{definition}

\begin{definition}[Direct product]
    A group $G$ is the \emph{direct product} of two of its invariant subgroups $G = G_1 \otimes G_2$ if
    \begin{enumerate}
        \item the only common element between $G_1$ and $G_2$ is the identity: $G_1 \cap G_2 = \{\idop_G\}$
        \item each element of $G_1$ commutes with each element of $G_2$: $g_1 g_2 = g_2 g_1 \forall g_1 \in G_1, g_2 \in G_2$
        \item every element of $g \in G$ admits a unique decomposition in terms of an element of $G_1$ and an element of $G_2$: $\forall g \in G, \exists ! g_1 \in G_1, g_2 \in G_2, g = g_1 g_2$
    \end{enumerate}
\end{definition}

\begin{definition}[Coset]
    Let $H$ be a subgroup of $G$ and $g \in G \setminus H$. We define the \emph{left coset} and \emph{right coset} of $H$ respectively
    \begin{align*}
        gH &= \{gh | h \in H\} \\
        Hg &= \{hg | h \in H\}
    \end{align*}
\end{definition}

\begin{proposition}
    Let $H$ be an invariant subgroup of $G$. Then, the left and right cosets of $H$ coincide $gH = Hg$.
\end{proposition}
\begin{proof}
    We need to show that $\forall l \in gH, l \in Hg$ and viceversa.

    Indeed, if $l \in gH$ then there exists $h \in H$ such that $l = gh$.
    Since $H$ is an invariant subgroup, it means that $ghg^{-1} \in H$, and therefore $h' = lg^{-1} \in H$. Now, since $h'g \in Hg$, we have $lg^{-1}g = l \in Hg$.
    
    Conversely, if $r \in Hg$, then there exists $h \in H$ such that $r = hg$.
    Since $H$ is an invariant subgroup, it means that $g^{-1}hg \in H$ and therefore $h' = g^{-1}r \in H$. Since $gh' \in gH$, we have $g^{-1}gr = r \in gH$.
\end{proof}

\begin{definition}[Quotient group]
    Given an invariant subgroup $H$ of a group $G$, we define the quotient group $G/H$ as the coset of $H$ alongside with the identity $G/H = gH \cup \{\idop H\} = Hg \cup \{\idop H\}$ with the multiplication law $\forall g_1, g_2 \in G \setminus H \cup \{\idop\}$
    \begin{equation*}
        g_1H \cdot g_2H = (g_1 \cdot g_2) H
    \end{equation*}
\end{definition}


\section{Homomorphisms}

\begin{definition}[Homomorphism]
    Let $(G, \cdot)$ and $(H, \ast)$ be two groups and $\phi: G \rightarrow H$.
    Then, we say that $\phi$ is a \emph{homomorphism} if
    \begin{equation*}
        \forall g_1, g_2 \in G, \phi(g_1 \cdot g_2) = \phi(g_1) \ast \phi(g_2)
    \end{equation*}
\end{definition}

\begin{definition}[Endomorphism]
    An \emph{endomorphism} is a homomorphism from a group $G$ to itself.
\end{definition}

\begin{definition}[Isomorphism]
    An \emph{isomorphism} is a bijective homomorphism.
\end{definition}
\begin{remark}
    Isomorphism between two groups is an equivalence relation.
\end{remark}

\begin{definition}[Automorphism]
    An \emph{automorphism} is a bijective endomorphism, or equivalently an isomorphism from a group $G$ to itself.
\end{definition}

\begin{definition}[Kernel]
    The kernel of a homomorphism $\phi: G \rightarrow H$ is the set $\ker{G}$ of all elements of $G$ that are mapped to the identity of $H$.
\end{definition}

\begin{proposition}
    The kernel $\ker{G}$ of a homomorphism $\phi: G \rightarrow H$ is an invariant subgroup of $G$.
\end{proposition}
\begin{proof}
    We need to show that $\forall k \in \ker{G}, \forall g \in G, gkg^{-1} \in \ker{G}$.
    Let us evaluate $\phi(gkg^{-1})$. By definition of homomorphism and by associativity we indeed have
    \begin{equation*}
        \phi(gkg^{-1}) = \phi(g) \phi(k) \phi(g)^{-1} = \phi(g) \idop_H \phi(g)^{-1} = \idop_H
    \end{equation*}
\end{proof}


\section{Classical groups}
This section references \cite{Zucchini}, chapters 2 and 3.

\begin{definition}[Function group]
    Let $A$ be a set. We define the function group $S(A)$ as the set of all invertible functions
    \begin{equation*}
        S(A) = \{f: A \rightarrow A | \exists f^{-1}\}
    \end{equation*}
\end{definition}
\begin{proposition}
    The set $S(A)$ endowed with the composition operation $(f_1 \circ f_2)(a) \equiv f_1(f_2(a))$ forms a group, with the identity being the function $Id(a) = a$.
\end{proposition}

\begin{definition}[$GL(V)$]
    Let $V$ be a finite-dimensional linear space over the field $\mathbb{K} \in \{\mathbb{R}, \mathbb{C}\}$. The \emph{general linear group} $GL(V)$ is defined as the set of all invertible endomorphisms in $V$.
    \begin{equation*}
        GL(V) = \text{End}(V) \cap S(V)
    \end{equation*}
\end{definition}
\begin{proposition}
    $GL(V)$ is a subgroup of $S(V)$, hence itself a group.
\end{proposition}

\begin{remark}
    Since all the elements in $GL(V)$ are invertible, then they have non-vanishing determinant.
\end{remark}


\begin{definition}[$SL(V)$]
    Let $V$ be a finite-dimensional linear space over the field $\mathbb{K} \in \{\mathbb{R}, \mathbb{C}\}$. We define the \emph{special linear group} as
    \begin{equation*}
        SL(V) = \{X \in GL(V) | \det X = 1\}
    \end{equation*}
\end{definition}
\begin{proposition}
    $SL(V)$ is a subgroup of $GL(V)$.
\end{proposition}





\section{Representations}


\begin{definition}[Representation]
    Let $G$ be a group and $V$ a linear space over real or complex scalars. A \emph{representation} of $G$ on $V$ is any group homomorphism $D: G \rightarrow \text{Aut}(V)$.
\end{definition}
\begin{remark}
    By definition of homomorphism, we have that $\forall g, h \in G$,
    \begin{enumerate}
        \item $D(gh) = D(g) D(h)$
        \item $D(g^{-1}) = D(g)^{-1}$
        \item $D(\idop_G) = \idop_V$
    \end{enumerate}
\end{remark}


\begin{proposition}[Direct sum of representations]
    Let $G$ be a group and $V_i$ with $i \in \{1, \ldots, N\}$ be linear spaces on the same field $\mathbb{K}$. Let also $D_i: G \rightarrow V_i$ be representations of $G$.
    Then, the mapping $R: G \rightarrow \text{Aut}(\oplus_i V_i)$ defined by
    \begin{equation*}
        R(g) = \oplus_i D_i(g)
    \end{equation*}
    for all $g \in G$, is a representation of $G$ in $\oplus_i V_i$, called \emph{direct sum} of the representations $D_i$ and indicated with $R = \oplus_i D_i$
\end{proposition}


\begin{proposition}[Direct product of representations]
    Let $G$ be a group and $V_i$ with $i \in \{1, \ldots, N\}$ be linear spaces on the same field $\mathbb{K}$. Let also $D_i: G \rightarrow V_i$ be representations of $G$.
    Then, the mapping $R: G \rightarrow \text{Aut}(\otimes_i V_i)$ defined by
    \begin{equation*}
        R(g) = \otimes_i D_i(g)
    \end{equation*}
    for all $g \in G$, is a representation of $G$ in $\otimes_i V_i$, called \emph{direct product} of the representations $D_i$ and indicated with $R = \otimes_i D_i$
\end{proposition}